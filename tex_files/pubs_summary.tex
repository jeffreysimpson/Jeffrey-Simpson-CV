74 refereed publications. 13 refeered publications as first author.

Total citations~=~3723; h-index~=~34 (2024-01-10)