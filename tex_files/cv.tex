%%%%%%%%%%%%%%%%%%%%%%%%%%%%%%%%%%%%%%%%%%%%%%%%%%%%%%%%%%%%%%%%%%%%%%%%
%%%%%%%%%%%%%%%%%%%%%% Simple LaTeX CV Template %%%%%%%%%%%%%%%%%%%%%%%%
%%%%%%%%%%%%%%%%%%%%%%%%%%%%%%%%%%%%%%%%%%%%%%%%%%%%%%%%%%%%%%%%%%%%%%%%

%%%%%%%%%%%%%%%%%%%%%%%%%%%%%%%%%%%%%%%%%%%%%%%%%%%%%%%%%%%%%%%%%%%%%%%%
%% NOTE: If you find that it says                                     %%
%%                                                                    %%
%%                           1 of ??                                  %%
%%                                                                    %%
%% at the bottom of your first page, this means that the AUX file     %%
%% was not available when you ran LaTeX on this source. Simply RERUN  %%
%% LaTeX to get the ``??'' replaced with the number of the last page  %%
%% of the document. The AUX file will be generated on the first run   %%
%% of LaTeX and used on the second run to fill in all of the          %%
%% references.                                                        %%
%%%%%%%%%%%%%%%%%%%%%%%%%%%%%%%%%%%%%%%%%%%%%%%%%%%%%%%%%%%%%%%%%%%%%%%%

%%%%%%%%%%%%%%%%%%%%%%%%%%%% Document Setup %%%%%%%%%%%%%%%%%%%%%%%%%%%%

% Don't like 10pt? Try 11pt or 12pt
\documentclass[10pt]{article}
\sloppy
\RequirePackage[T1]{fontenc}
% LaTeX will typeset using Computer Modern Roman, which a lot of
% non-mathematicians and non-engineers won't like. Also, a few PDF
% viewers may not render CMR very well. Instead, Times New Roman can
% be used. That's what this package does.
\usepackage{palatino}
\usepackage{fancyhdr,lastpage}
\usepackage{color,hyperref}
% This is a helpful package that puts math inside length specifications
\usepackage{calc}
% This package helps LaTeX auto-hyphenate hyphenated words if you use
% special hyphens. For example, bio\-/mimicry will properly hyphenate
% ``mimicry'' if necessary.
\usepackage[shortcuts]{extdash}
% Provides special list environments and macros to create new ones
\usepackage[shortlabels]{enumitem}

\hyphenation{citations}
\hyphenation{MNRAS}

% Layout: Puts the section titles on left side of page
\reversemarginpar

% Text formatting.
\newcommand{\foreign}[1]{\textit{#1}}
\newcommand{\etal}{\foreign{et~al.}}
\newcommand{\project}[1]{\textsl{#1}}
\definecolor{grey}{rgb}{0.5,0.5,0.5}
\newcommand{\deemph}[1]{\textcolor{grey}{\footnotesize{#1}}}

%
%         PAPER SIZE, PAGE NUMBER, AND DOCUMENT LAYOUT NOTES:
%
% The next \usepackage line changes the layout for CV style section
% headings as marginal notes. It also sets up the paper size as either
% letter or A4. By default, letter was used. If A4 paper is desired,
% comment out the letterpaper lines and uncomment the a4paper lines.
%
% As you can see, the margin widths and section title widths can be
% easily adjusted.
%
% ALSO: Notice that the includefoot option can be commented OUT in order
% to put the PAGE NUMBER *IN* the bottom margin. This will make the
% effective text area larger.
%
% IF YOU WISH TO REMOVE THE ``of LASTPAGE'' next to each page number,
% see the note about the +LP and -LP lines below. Comment out the +LP
% and uncomment the -LP.
%
% IF YOU WISH TO REMOVE PAGE NUMBERS, be sure that the includefoot line
% is uncommented and ALSO uncomment the \pagestyle{empty} a few lines
% below.
%

%% Use these lines for letter-sized paper
%\usepackage[paper=letterpaper,
%            %includefoot, % Uncomment to put page number above margin
%            marginparwidth=1.2in,     % Length of section titles
%            marginparsep=.05in,       % Space between titles and text
%            margin=1in,               % 1 inch margins
%            includemp]{geometry}

%% Use these lines for A4-sized paper
\usepackage[paper=a4paper,
            %includefoot, % Uncomment to put page number above margin
            marginparwidth=0.mm,    % Length of section titles
            marginparsep=1.5mm,       % Space between titles and text
            margin=25mm,              % 25mm margins
            includemp]{geometry}

%% More layout: Get rid of indenting throughout entire document
\setlength{\parindent}{0in}

% Set up the custom unordered list.
\newcounter{refpubnum}
\newcommand{\cvlist}{}

\definecolor{numcolor}{rgb}{0.5,0.5,0.5}

\usepackage{xpatch}%
\newcommand\mnras{MNRAS}
\newcommand\mnrasl{MNRASL}
\newcommand\apjs{ApJS}
\newcommand\apj{ApJ}
\newcommand\aj{AJ}                   % Astronomical Journal (the)
\newcommand{\doi}[2]{{\href{http://dx.doi.org/#1}{{#2}}}}
\newcommand{\ads}[2]{\href{http://adsabs.harvard.edu/abs/#1}{{#2}}}
\newcommand{\isbn}[1]{{\footnotesize(\textsc{isbn:}{#1})}}
\newcommand{\arxiv}[1]{{\href{http://arxiv.org/abs/#1}{arXiv:{#1}}}}


\pagestyle{fancy}
%\pagestyle{empty}      % Uncomment this to get rid of page numbers
\fancyhf{}\renewcommand{\headrulewidth}{0pt}
\fancyfootoffset{\marginparsep+\marginparwidth}
\newlength{\footpageshift}
\setlength{\footpageshift}
          {0.5\textwidth+0.5\marginparsep+0.5\marginparwidth-2in}
\lfoot{\hspace{\footpageshift}%
       \parbox{4in}{\, \hfill %
                    \arabic{page} of \protect\pageref*{LastPage} % +LP
%                    \arabic{page}                               % -LP
                    \hfill \,}}

% Finally, give us PDF bookmarks

\definecolor{darkblue}{rgb}{0.0,0.0,0.3}
\hypersetup{colorlinks,breaklinks,
            linkcolor=darkblue,urlcolor=darkblue,
            anchorcolor=darkblue,citecolor=darkblue}

%%%%%%%%%%%%%%%%%%%%%%%% End Document Setup %%%%%%%%%%%%%%%%%%%%%%%%%%%%


%%%%%%%%%%%%%%%%%%%%%%%%%%% Helper Commands %%%%%%%%%%%%%%%%%%%%%%%%%%%%

%%% HEADING AT TOP OF CURRICULUM VITAE

% The title (name) with a horizontal rule under it
% (optional argument typesets an object right-justified across from name
%  as well)
%
% Usage: \makeheading{name}
%        OR
%        \makeheading[right_object]{name}
%
% Place at top of document. It should be the first thing.
% If ``right_object'' is provided in the square-braced optional
% argument, it will be right justified on the same line as ``name'' at
% the top of the CV. For example:
%
%       \makeheading[\emph{Curriculum vitae}]{Your Name}
%
% will put an emphasized ``Curriculum vitae'' at the top of the document
% as a title. Likewise, a picture could be included:
%
%   \makeheading[{\includegraphics[height=1.5in]{my_picture}}]{Your Name}
%
% the picture will be flush right across from the name. For this example
% to work, make sure the extra set of curly braces is included. Also
% makes ure that \usepackage{graphicx} is somewhere in the preamble.
\newcommand{\makeheading}[2][]%
        {\hspace*{-\marginparsep minus \marginparwidth}%
         \begin{minipage}[t]{\textwidth+\marginparwidth+\marginparsep}%
             {\large \bfseries #2 \hfill #1}\\[-0.15\baselineskip]%
                 \rule{\columnwidth}{1pt}%
         \end{minipage}}

%%% SECTION HEADINGS

% The section headings. Flush left in small caps down pseudo-margin.
%
% Usage: \subsection*{section name}
%\renewcommand{\subsection*}[1]{\pagebreak[3]%
%    \vspace{1.3\baselineskip}%
%    \phantomsection\addcontentsline{toc}{section}{#1}%
%    \noindent\llap{\scshape\smash{\parbox[t]{\marginparwidth}{\hyphenpenalty=10000\raggedright #1}}}%
%    \vspace{-\baselineskip}\par}

%%% LISTS

% This macro alters a list by removing some of the space that follows the list
% (is used by lists below)
\newcommand*\fixendlist[1]{%
    \expandafter\let\csname preFixEndListend#1\expandafter\endcsname\csname end#1\endcsname
    \expandafter\def\csname end#1\endcsname{\csname preFixEndListend#1\endcsname\vspace{-0.6\baselineskip}}}

% These macros help ensure that items in outer-type lists do not get
% separated from the next line by a page break
% (they are used by lists below)
\let\originalItem\item
\newcommand*\fixouterlist[1]{%
    \expandafter\let\csname preFixOuterList#1\expandafter\endcsname\csname #1\endcsname
    \expandafter\def\csname #1\endcsname{\let\oldItem\item\def\item{\pagebreak[2]\oldItem}\csname preFixOuterList#1\endcsname}
    \expandafter\let\csname preFixOuterListend#1\expandafter\endcsname\csname end#1\endcsname
    \expandafter\def\csname end#1\endcsname{\let\item\oldItem\csname preFixOuterListend#1\endcsname}}
\newcommand*\fixinnerlist[1]{%
    \expandafter\let\csname preFixInnerList#1\expandafter\endcsname\csname #1\endcsname
    \expandafter\def\csname #1\endcsname{\let\oldItem\item\let\item\originalItem\csname preFixInnerList#1\endcsname}
    \expandafter\let\csname preFixInnerListend#1\expandafter\endcsname\csname end#1\endcsname
    \expandafter\def\csname end#1\endcsname{\csname preFixInnerListend#1\endcsname\let\item\oldItem}}

% An itemize-style list with lots of space between items
%
% Usage:
%   \begin{outerlist}
%       \item ...    % (or \item[] for no bullet)
%   \end{outerlist}
\newlist{outerlist}{itemize}{3}
    \setlist[outerlist]{label=\enskip\textbullet,leftmargin=*}
    \fixendlist{outerlist}
    \fixouterlist{outerlist}

% An environment IDENTICAL to outerlist that has better pre-list spacing
% when used as the first thing in a \subsection*
%
% Usage:
%   \begin{lonelist}
%       \item ...    % (or \item[] for no bullet)
%   \end{lonelist}
\newlist{lonelist}{itemize}{3}
    \setlist[lonelist]{label=\enskip\textbullet,leftmargin=*,partopsep=0pt,topsep=0pt}
    \fixendlist{lonelist}
    \fixouterlist{lonelist}

% An itemize-style list with little space between items
%
% Usage:
%   \begin{innerlist}
%       \item ...    % (or \item[] for no bullet)
%   \end{innerlist}
\newlist{innerlist}{itemize}{3}
    \setlist[innerlist]{label=\enskip\textbullet,leftmargin=*,parsep=0pt,itemsep=0pt,topsep=0pt,partopsep=0pt}
    \fixinnerlist{innerlist}

% An environment IDENTICAL to innerlist that has better pre-list spacing
% when used as the first thing in a \subsection*
%
% Usage:
%   \begin{loneinnerlist}
%       \item ...    % (or \item[] for no bullet)
%   \end{loneinnerlist}
\newlist{loneinnerlist}{itemize}{3}
    \setlist[loneinnerlist]{label=\enskip\textbullet,leftmargin=*,parsep=0pt,itemsep=0pt,topsep=0pt,partopsep=0pt}
    \fixendlist{loneinnerlist}
    \fixinnerlist{loneinnerlist}

%%% EXTRA SPACE

% To add some paragraph space between lines.
% This also tells LaTeX to preferably break a page on one of these gaps
% if there is a needed pagebreak nearby.
\newcommand{\blankline}{\quad\pagebreak[3]}
\newcommand{\halfblankline}{\quad\vspace{-0.5\baselineskip}\pagebreak[3]}

%%% FORMATTING MACROS
\usepackage{url}

% You can adjust the style \url{} uses here:
% (options are: same, rm, sf, tt; defaults to tt)
\urlstyle{same}

% For \email{ADDRESS}, links ADDRESS to the url mailto:ADDRESS
% (uncomment to typeset the e\-/mail address in typewriter font;
%  otherwise, will be typeset in the \urlstyle above)
%\DeclareUrlCommand\emaillink{\urlstyle{tt}}
\providecommand*\emaillink[1]{\nolinkurl{#1}}
\providecommand*\email[1]{\href{mailto:#1}{\emaillink{#1}}}

\providecommand\BibTeX{{B\kern-.05em{\sc i\kern-.025em b}\kern-.08em \TeX}}


%%%%%%%%%%%%%%%%%%%%%%%% End Helper Commands %%%%%%%%%%%%%%%%%%%%%%%%%%%

%%%%%%%%%%%%%%%%%%%%%%%%% Begin CV Document %%%%%%%%%%%%%%%%%%%%%%%%%%%%

\begin{document}
\makeheading{Dr.~Jeffrey~D.~Simpson\\Curriculum vitae}

\subsection*{Contact Information}

% NOTE: Mind where the & separators and \\ breaks are in the following
%       table. Table is one row made up of three parboxes. The left
%       parbox has address info, the middle parbox has a vertical bar,
%       and the right parbox has phone and electronic contact
%       information.
%
% MACROS: \rcollength is the width of the right column of the table
%             (adjust it to your liking; default is 1.85in).
%         \spacewidth is width of area between left and right boxes.
%
\newlength{\rcollength}\setlength{\rcollength}{2.3in}%
\newlength{\spacewidth}\setlength{\spacewidth}{10pt}
%
\begin{tabular}[t]{@{}p{\textwidth-\rcollength-\spacewidth}@{}p{\spacewidth}@{}p{\rcollength}}%

% Address box
\parbox{\textwidth-\rcollength-\spacewidth}{%
School of Physics\\
\href{http://www.unsw.edu.au/}{University of New South Wales}\\
Sydney NSW Australia}

&
% Uncomment to add a vertical bar in middle of contact information
% {\vrule width 0.5pt}
\parbox[m][5\baselineskip]{\spacewidth}{} &

% Non-snail-mail contact information
\parbox{\rcollength}{%
% (02) 9372 4838 \\
\email{jeffrey.simpson@unsw.edu.au}}
\end{tabular}

\subsection*{Education}

\textbf{Ph.D. in Astronomy},
\href{http://www.canterbury.ac.nz/}{University of Canterbury}
New Zealand,
             \hfill 2014

\textbf{M.Sc. in Astronomy},
\href{http://www.canterbury.ac.nz/}{University of Canterbury}
New Zealand,
             \hfill 2009

\subsection*{Current Employment}

\textbf{Post-Doctoral Research Fellow},
            University of New South Wales
            \hfill {2018}

\vspace{0.1in}

\subsection*{Previous Employment}

\textbf{Research Fellow},
            Australian Astronomical Observatory
            \hfill {2015 to 2018}

\textbf{Research Fellow},
            Macquarie University
            \hfill {2013 to 2015}

\vspace{0.1in}

\subsection*{Refereed Publications}
73 refereed publications. 13 refeered publications as first author.

Total citations~=~2686; h-index~=~27 (2022-07-07)
\begin{list}{}{\cvlist}
\item[{\color{numcolor}\scriptsize64}] Casagrande, Luca, Lin, Jane, Rains, Adam D., \etal\ (incl.\ \textbf{JDS}), 2021, \doi{10.1093/mnras/stab2304}{The GALAH survey: effective temperature calibration from the InfraRed Flux Method in the Gaia system}, \mnras, \textbf{507}, 2684 [\href{https://ui.adsabs.harvard.edu/#abs/2021MNRAS.507.2684C}{12~citations}]

\item[{\color{numcolor}\scriptsize63}] \textbf{Simpson, Jeffrey D.}, Martell, Sarah L., Buder, Sven, \etal, 2021, \doi{10.1093/mnras/stab2012}{The GALAH survey: accreted stars also inhabit the Spite plateau}, \mnras, \textbf{507}, 43 [\href{https://ui.adsabs.harvard.edu/#abs/2021MNRAS.507...43S}{8~citations}]

\item[{\color{numcolor}\scriptsize62}] Buder, Sven, Sharma, Sanjib, Kos, Janez, \etal\ (incl.\ \textbf{JDS}), 2021, \doi{10.1093/mnras/stab1242}{The GALAH+ survey: Third data release}, \mnras, \textbf{506}, 150 [\href{https://ui.adsabs.harvard.edu/#abs/2021MNRAS.506..150B}{65~citations}]

\item[{\color{numcolor}\scriptsize61}] Sharma, Sanjib, Hayden, Michael R., Bland-Hawthorn, Joss, \etal\ (incl.\ \textbf{JDS}), 2021, \doi{10.1093/mnras/stab1086}{Fundamental relations for the velocity dispersion of stars in the Milky Way}, \mnras, \textbf{506}, 1761 [\href{https://ui.adsabs.harvard.edu/#abs/2021MNRAS.506.1761S}{19~citations}]

\item[{\color{numcolor}\scriptsize60}] Kos, Janez, Bland-Hawthorn, Joss, Buder, Sven, \etal\ (incl.\ \textbf{JDS}), 2021, \doi{10.1093/mnras/stab1767}{The GALAH survey: Chemical homogeneity of the Orion complex}, \mnras, \textbf{506}, 4232 [\href{https://ui.adsabs.harvard.edu/#abs/2021MNRAS.506.4232K}{3~citations}]

\item[{\color{numcolor}\scriptsize59}] Zwitter, Toma{\v{z}}, Kos, Janez, Buder, Sven, \etal\ (incl.\ \textbf{JDS}), 2021, \doi{10.1093/mnras/stab2673}{The GALAH+ Survey: A new library of observed stellar spectra improves radial velocities and hints at motions within M67}, \mnras [\href{https://ui.adsabs.harvard.edu/#abs/2021MNRAS.tmp.2413Z}{2~citations}]

\item[{\color{numcolor}\scriptsize58}] Martell, Sarah L., \textbf{Simpson, Jeffrey D.}, Balasubramaniam, Adithya G., \etal, 2021, \doi{10.1093/mnras/stab1356}{The GALAH survey: a census of lithium-rich giant stars}, \mnras, \textbf{505}, 5340 [\href{https://ui.adsabs.harvard.edu/#abs/2021MNRAS.505.5340M}{16~citations}]

\item[{\color{numcolor}\scriptsize57}] Munari, U., Traven, G., Masetti, N., \etal\ (incl.\ \textbf{JDS}), 2021, \doi{10.1093/mnras/stab1620}{The GALAH survey and symbiotic stars - I. Discovery and follow-up of 33 candidate accreting-only systems}, \mnras, \textbf{505}, 6121 [\href{https://ui.adsabs.harvard.edu/#abs/2021MNRAS.505.6121M}{1~citation}]

\item[{\color{numcolor}\scriptsize56}] Clark, Jake T., Clert{\'e}, Mathieu, Hinkel, Natalie R., \etal\ (incl.\ \textbf{JDS}), 2021, \doi{10.1093/mnras/stab1052}{The GALAH Survey: using galactic archaeology to refine our knowledge of TESS target stars}, \mnras, \textbf{504}, 4968 [\href{https://ui.adsabs.harvard.edu/#abs/2021MNRAS.504.4968C}{2~citations}]

\item[{\color{numcolor}\scriptsize55}] Hansen, Terese T., Ji, Alexander P., Da Costa, Gary S., \etal\ (incl.\ \textbf{JDS}), 2021, \doi{10.3847/1538-4357/abfc54}{S$^5$: The Destruction of a Bright Dwarf Galaxy as Revealed by the Chemistry of the Indus Stellar Stream}, \apj, \textbf{915}, 103

\item[{\color{numcolor}\scriptsize54}] Spina, L., Ting, Y. -S., De Silva, G. M., \etal\ (incl.\ \textbf{JDS}), 2021, \doi{10.1093/mnras/stab471}{The GALAH survey: tracing the Galactic disc with open clusters}, \mnras, \textbf{503}, 3279 [\href{https://ui.adsabs.harvard.edu/#abs/2021MNRAS.503.3279S}{12~citations}]

\item[{\color{numcolor}\scriptsize53}] Zucker, Daniel B., \textbf{Simpson, Jeffrey D.}, Martell, Sarah L., \etal, 2021, \doi{10.3847/2041-8213/abf7cd}{The GALAH Survey: No Chemical Evidence of an Extragalactic Origin for the Nyx Stream}, \apj, \textbf{912} [\href{https://ui.adsabs.harvard.edu/#abs/2021ApJ...912L..30Z}{1~citation}]

\item[{\color{numcolor}\scriptsize52}] Li, Ting S., Koposov, Sergey E., Erkal, Denis, \etal\ (incl.\ \textbf{JDS}), 2021, \doi{10.3847/1538-4357/abeb18}{Broken into Pieces: ATLAS and Aliqa Uma as One Single Stream}, \apj, \textbf{911}, 149 [\href{https://ui.adsabs.harvard.edu/#abs/2021ApJ...911..149L}{17~citations}]

\item[{\color{numcolor}\scriptsize51}] Amarsi, A. M., Lind, K., Osorio, Y., \etal\ (incl.\ \textbf{JDS}), 2020, \doi{10.1051/0004-6361/202038650}{The GALAH Survey: non-LTE departure coefficients for large spectroscopic surveys}, Astronomy and Astrophysics, \textbf{642} [\href{https://ui.adsabs.harvard.edu/#abs/2020A&A...642A..62A}{25~citations}]

\item[{\color{numcolor}\scriptsize50}] Ji, Alexander P., Li, Ting S., Hansen, Terese T., \etal\ (incl.\ \textbf{JDS}), 2020, \doi{10.3847/1538-3881/abacb6}{The Southern Stellar Stream Spectroscopic Survey (S$^5$): Chemical Abundances of Seven Stellar Streams}, \aj, \textbf{160}, 181 [\href{https://ui.adsabs.harvard.edu/#abs/2020AJ....160..181J}{19~citations}]

\item[{\color{numcolor}\scriptsize49}] Gao, Xudong, Lind, Karin, Amarsi, Anish M., \etal\ (incl.\ \textbf{JDS}), 2020, \doi{10.1093/mnrasl/slaa109}{The GALAH survey: a new constraint on cosmological lithium and Galactic lithium evolution from warm dwarf stars}, \mnras, \textbf{497} [\href{https://ui.adsabs.harvard.edu/#abs/2020MNRAS.497L..30G}{12~citations}]

\item[{\color{numcolor}\scriptsize48}] Arentsen, Anke, Starkenburg, Else, Martin, Nicolas F., \etal\ (incl.\ \textbf{JDS}), 2020, \doi{10.1093/mnras/staa1661}{The Pristine Inner Galaxy Survey (PIGS) II: Uncovering the most metal-poor populations in the inner Milky Way}, \mnras, \textbf{496}, 4964 [\href{https://ui.adsabs.harvard.edu/#abs/2020MNRAS.496.4964A}{11~citations}]

\item[{\color{numcolor}\scriptsize47}] Wan, Zhen, Lewis, Geraint F., Li, Ting S., \etal\ (incl.\ \textbf{JDS}), 2020, \doi{10.1038/s41586-020-2483-6}{The tidal remnant of an unusually metal-poor globular cluster}, Nature, \textbf{583}, 768 [\href{https://ui.adsabs.harvard.edu/#abs/2020Natur.583..768W}{20~citations}]

\item[{\color{numcolor}\scriptsize46}] Wheeler, Adam, Ness, Melissa, Buder, Sven, \etal\ (incl.\ \textbf{JDS}), 2020, \doi{10.3847/1538-4357/ab9a46}{Abundances in the Milky Way across Five Nucleosynthetic Channels from 4 Million LAMOST Stars}, \apj, \textbf{898}, 58 [\href{https://ui.adsabs.harvard.edu/#abs/2020ApJ...898...58W}{17~citations}]

\item[{\color{numcolor}\scriptsize45}] Wittenmyer, Robert A., Clark, Jake T., Sharma, Sanjib, \etal\ (incl.\ \textbf{JDS}), 2020, \doi{10.1093/mnras/staa1528}{K2-HERMES II. Planet-candidate properties from K2 Campaigns 1-13}, \mnras, \textbf{496}, 851 [\href{https://ui.adsabs.harvard.edu/#abs/2020MNRAS.496..851W}{5~citations}]

\item[{\color{numcolor}\scriptsize44}] Kawka, Adela, \textbf{Simpson, Jeffrey D.}, Vennes, St{\'e}phane, \etal, 2020, \doi{10.1093/mnrasl/slaa068}{The closest extremely low-mass white dwarf to the Sun}, \mnras, \textbf{495} [\href{https://ui.adsabs.harvard.edu/#abs/2020MNRAS.495L.129K}{6~citations}]

\item[{\color{numcolor}\scriptsize43}] Traven, G., Feltzing, S., Merle, T., \etal\ (incl.\ \textbf{JDS}), 2020, \doi{10.1051/0004-6361/202037484}{The GALAH survey: multiple stars and our Galaxy. I. A comprehensive method for deriving properties of FGK binary stars}, Astronomy and Astrophysics, \textbf{638} [\href{https://ui.adsabs.harvard.edu/#abs/2020A&A...638A.145T}{9~citations}]

\item[{\color{numcolor}\scriptsize42}] \textbf{Simpson, Jeffrey D.}, 2020, \doi{10.3847/2515-5172/ab917e}{Empirical Relationship between Calcium Triplet Equivalent Widths and [Fe/H] Using Gaia Photometry}, Research Notes of the American Astronomical Society, \textbf{4}, 70 [\href{https://ui.adsabs.harvard.edu/#abs/2020RNAAS...4...70S}{1~citation}]

\item[{\color{numcolor}\scriptsize41}] Borsato, Nicholas W., Martell, Sarah L., \& \textbf{Simpson, Jeffrey D.}, 2020, \doi{10.1093/mnras/stz3479}{Identifying stellar streams in Gaia DR2 with data mining techniques}, \mnras, \textbf{492}, 1370 [\href{https://ui.adsabs.harvard.edu/#abs/2020MNRAS.492.1370B}{16~citations}]

\item[{\color{numcolor}\scriptsize40}] Koposov, Sergey E., Boubert, Douglas, Li, Ting S., \etal\ (incl.\ \textbf{JDS}), 2020, \doi{10.1093/mnras/stz3081}{Discovery of a nearby 1700 km s$^{-1}$ star ejected from the Milky Way by Sgr A*}, \mnras, \textbf{491}, 2465 [\href{https://ui.adsabs.harvard.edu/#abs/2020MNRAS.491.2465K}{42~citations}]

\item[{\color{numcolor}\scriptsize39}] Lin, Jane, Asplund, Martin, Ting, Yuan-Sen, \etal\ (incl.\ \textbf{JDS}), 2020, \doi{10.1093/mnras/stz3048}{The GALAH survey: temporal chemical enrichment of the galactic disc}, \mnras, \textbf{491}, 2043 [\href{https://ui.adsabs.harvard.edu/#abs/2020MNRAS.491.2043L}{17~citations}]

\item[{\color{numcolor}\scriptsize38}] Arentsen, A., Starkenburg, E., Martin, N. F., \etal\ (incl.\ \textbf{JDS}), 2020, \doi{10.1093/mnrasl/slz156}{The Pristine Inner Galaxy Survey (PIGS) I: tracing the kinematics of metal-poor stars in the Galactic bulge}, \mnras, \textbf{491} [\href{https://ui.adsabs.harvard.edu/#abs/2020MNRAS.491L..11A}{17~citations}]

\item[{\color{numcolor}\scriptsize37}] \textbf{Simpson, Jeffrey D.}, Martell, Sarah L., Da Costa, Gary, \etal, 2020, \doi{10.1093/mnras/stz3105}{The GALAH Survey: Chemically tagging the Fimbulthul stream to the globular cluster {\ensuremath{\omega}} Centauri}, \mnras, \textbf{491}, 3374 [\href{https://ui.adsabs.harvard.edu/#abs/2020MNRAS.491.3374S}{11~citations}]

\item[{\color{numcolor}\scriptsize36}] Sharma, Sanjib, Stello, Dennis, Bland-Hawthorn, Joss, \etal\ (incl.\ \textbf{JDS}), 2019, \doi{10.1093/mnras/stz2861}{The K2-HERMES Survey: age and metallicity of the thick disc}, \mnras, \textbf{490}, 5335 [\href{https://ui.adsabs.harvard.edu/#abs/2019MNRAS.490.5335S}{41~citations}]

\item[{\color{numcolor}\scriptsize35}] Li, T. S., Koposov, S. E., Zucker, D. B., \etal\ (incl.\ \textbf{JDS}), 2019, \doi{10.1093/mnras/stz2731}{The southern stellar stream spectroscopic survey (S$^5$): Overview, target selection, data reduction, validation, and early science}, \mnras, \textbf{490}, 3508 [\href{https://ui.adsabs.harvard.edu/#abs/2019MNRAS.490.3508L}{43~citations}]

\item[{\color{numcolor}\scriptsize34}] Casey, Andrew R., Lattanzio, John C., Aleti, Aldeida, \etal\ (incl.\ \textbf{JDS}), 2019, \doi{10.3847/1538-4357/ab4fea}{A Data-driven Model of Nucleosynthesis with Chemical Tagging in a Lower-dimensional Latent Space}, \apj, \textbf{887}, 73 [\href{https://ui.adsabs.harvard.edu/#abs/2019ApJ...887...73C}{6~citations}]

\item[{\color{numcolor}\scriptsize33}] Khanna, Shourya, Sharma, Sanjib, Tepper-Garcia, Thor, \etal\ (incl.\ \textbf{JDS}), 2019, \doi{10.1093/mnras/stz2462}{The GALAH survey and Gaia DR2: Linking ridges, arches, and vertical waves in the kinematics of the Milky Way}, \mnras, \textbf{489}, 4962 [\href{https://ui.adsabs.harvard.edu/#abs/2019MNRAS.489.4962K}{40~citations}]

\item[{\color{numcolor}\scriptsize32}] Shipp, N., Li, T. S., Pace, A. B., \etal\ (incl.\ \textbf{JDS}), 2019, \doi{10.3847/1538-4357/ab44bf}{Proper Motions of Stellar Streams Discovered in the Dark Energy Survey}, \apj, \textbf{885}, 3 [\href{https://ui.adsabs.harvard.edu/#abs/2019ApJ...885....3S}{29~citations}]

\item[{\color{numcolor}\scriptsize31}] \textbf{Simpson, Jeffrey D.}, \& Martell, Sarah L., 2019, \doi{10.1093/mnras/stz2611}{A nitrogen-enhanced metal-poor star discovered in the globular cluster ESO280-SC06}, \mnras, \textbf{490}, 741 [\href{https://ui.adsabs.harvard.edu/#abs/2019MNRAS.490..741S}{7~citations}]

\item[{\color{numcolor}\scriptsize30}] Kos, Janez, Bland-Hawthorn, Joss, Asplund, Martin, \etal\ (incl.\ \textbf{JDS}), 2019, \doi{10.1051/0004-6361/201834710}{Discovery of a 21 Myr old stellar population in the Orion complex{\ensuremath{\star}}}, Astronomy and Astrophysics, \textbf{631} [\href{https://ui.adsabs.harvard.edu/#abs/2019A&A...631A.166K}{15~citations}]

\item[{\color{numcolor}\scriptsize29}] \textbf{Simpson, Jeffrey D.}, 2019, \doi{10.1093/mnras/stz1699}{The retrograde orbit of the globular cluster FSR1758 revealed with Gaia DR2}, \mnras, \textbf{488}, 253 [\href{https://ui.adsabs.harvard.edu/#abs/2019MNRAS.488..253S}{9~citations}]

\item[{\color{numcolor}\scriptsize28}] {\v{C}}otar, Klemen, Zwitter, Toma{\v{z}}, Traven, Gregor, \etal\ (incl.\ \textbf{JDS}), 2019, \doi{10.1093/mnras/stz1397}{The GALAH survey: unresolved triple Sun-like stars discovered by the Gaia mission}, \mnras, \textbf{487}, 2474 [\href{https://ui.adsabs.harvard.edu/#abs/2019MNRAS.487.2474C}{3~citations}]

\item[{\color{numcolor}\scriptsize27}] Bland-Hawthorn, Joss, Sharma, Sanjib, Tepper-Garcia, Thor, \etal\ (incl.\ \textbf{JDS}), 2019, \doi{10.1093/mnras/stz217}{The GALAH survey and Gaia DR2: dissecting the stellar disc's phase space by age, action, chemistry, and location}, \mnras, \textbf{486}, 1167 [\href{https://ui.adsabs.harvard.edu/#abs/2019MNRAS.486.1167B}{101~citations}]

\item[{\color{numcolor}\scriptsize26}] Buder, S., Lind, K., Ness, M. K., \etal\ (incl.\ \textbf{JDS}), 2019, \doi{10.1051/0004-6361/201833218}{The GALAH survey: An abundance, age, and kinematic inventory of the solar neighbourhood made with TGAS}, Astronomy and Astrophysics, \textbf{624} [\href{https://ui.adsabs.harvard.edu/#abs/2019A&A...624A..19B}{69~citations}]

\item[{\color{numcolor}\scriptsize25}] \textbf{Simpson, Jeffrey D.}, Martell, Sarah L., Da Costa, Gary, \etal, 2019, \doi{10.1093/mnras/sty3042}{The GALAH survey: co-orbiting stars and chemical tagging}, \mnras, \textbf{482}, 5302 [\href{https://ui.adsabs.harvard.edu/#abs/2019MNRAS.482.5302S}{12~citations}]

\item[{\color{numcolor}\scriptsize24}] Khanna, Shourya, Sharma, Sanjib, Bland-Hawthorn, Joss, \etal\ (incl.\ \textbf{JDS}), 2019, \doi{10.1093/mnras/sty2924}{The GALAH survey: velocity fluctuations in the Milky Way using Red Clump giants}, \mnras, \textbf{482}, 4215 [\href{https://ui.adsabs.harvard.edu/#abs/2019MNRAS.482.4215K}{4~citations}]

\item[{\color{numcolor}\scriptsize23}] Gao, Xudong, Lind, Karin, Amarsi, Anish M., \etal\ (incl.\ \textbf{JDS}), 2018, \doi{10.1093/mnras/sty2414}{The GALAH survey: verifying abundance trends in the open cluster M67 using non-LTE modelling}, \mnras, \textbf{481}, 2666 [\href{https://ui.adsabs.harvard.edu/#abs/2018MNRAS.481.2666G}{33~citations}]

\item[{\color{numcolor}\scriptsize22}] Zwitter, Toma{\v{z}}, Kos, Janez, Chiavassa, Andrea, \etal\ (incl.\ \textbf{JDS}), 2018, \doi{10.1093/mnras/sty2293}{The GALAH survey: accurate radial velocities and library of observed stellar template spectra}, \mnras, \textbf{481}, 645 [\href{https://ui.adsabs.harvard.edu/#abs/2018MNRAS.481..645Z}{23~citations}]

\item[{\color{numcolor}\scriptsize21}] Kos, Janez, de Silva, Gayandhi, Buder, Sven, \etal\ (incl.\ \textbf{JDS}), 2018, \doi{10.1093/mnras/sty2171}{The GALAH survey and Gaia DR2: (non-)existence of five sparse high-latitude open clusters}, \mnras, \textbf{480}, 5242 [\href{https://ui.adsabs.harvard.edu/#abs/2018MNRAS.480.5242K}{20~citations}]

\item[{\color{numcolor}\scriptsize20}] Kos, Janez, Bland-Hawthorn, Joss, Betters, Christopher H., \etal\ (incl.\ \textbf{JDS}), 2018, \doi{10.1093/mnras/sty2175}{Holistic spectroscopy: complete reconstruction of a wide-field, multiobject spectroscopic image using a photonic comb}, \mnras, \textbf{480}, 5475 [\href{https://ui.adsabs.harvard.edu/#abs/2018MNRAS.480.5475K}{10~citations}]

\item[{\color{numcolor}\scriptsize19}] Buder, Sven, Asplund, Martin, Duong, Ly, \etal\ (incl.\ \textbf{JDS}), 2018, \doi{10.1093/mnras/sty1281}{The GALAH Survey: second data release}, \mnras, \textbf{478}, 4513 [\href{https://ui.adsabs.harvard.edu/#abs/2018MNRAS.478.4513B}{214~citations}]

\item[{\color{numcolor}\scriptsize18}] \textbf{Simpson, Jeffrey D.}, 2018, \doi{10.1093/mnras/sty847}{The most metal-poor Galactic globular cluster: the first spectroscopic observations of ESO280-SC06}, \mnras, \textbf{477}, 4565 [\href{https://ui.adsabs.harvard.edu/#abs/2018MNRAS.477.4565S}{15~citations}]

\item[{\color{numcolor}\scriptsize17}] Quillen, Alice C., De Silva, Gayandhi, Sharma, Sanjib, \etal\ (incl.\ \textbf{JDS}), 2018, \doi{10.1093/mnras/sty865}{The GALAH survey: stellar streams and how stellar velocity distributions vary with Galactic longitude, hemisphere, and metallicity}, \mnras, \textbf{478}, 228 [\href{https://ui.adsabs.harvard.edu/#abs/2018MNRAS.478..228Q}{27~citations}]

\item[{\color{numcolor}\scriptsize16}] Duong, L., Freeman, K. C., Asplund, M., \etal\ (incl.\ \textbf{JDS}), 2018, \doi{10.1093/mnras/sty525}{The GALAH survey: properties of the Galactic disc(s) in the solar neighbourhood}, \mnras, \textbf{476}, 5216 [\href{https://ui.adsabs.harvard.edu/#abs/2018MNRAS.476.5216D}{31~citations}]

\item[{\color{numcolor}\scriptsize15}] Kos, Janez, Bland-Hawthorn, Joss, Freeman, Ken, \etal\ (incl.\ \textbf{JDS}), 2018, \doi{10.1093/mnras/stx2637}{The GALAH survey: chemical tagging of star clusters and new members in the Pleiades}, \mnras, \textbf{473}, 4612 [\href{https://ui.adsabs.harvard.edu/#abs/2018MNRAS.473.4612K}{29~citations}]

\item[{\color{numcolor}\scriptsize14}] Wittenmyer, Robert A., Sharma, Sanjib, Stello, Dennis, \etal\ (incl.\ \textbf{JDS}), 2018, \doi{10.3847/1538-3881/aaa3e4}{The K2-HERMES Survey. I. Planet-candidate Properties from K2 Campaigns 1-3}, \aj, \textbf{155}, 84 [\href{https://ui.adsabs.harvard.edu/#abs/2018AJ....155...84W}{36~citations}]

\item[{\color{numcolor}\scriptsize13}] Sharma, Sanjib, Stello, Dennis, Buder, Sven, \etal\ (incl.\ \textbf{JDS}), 2018, \doi{10.1093/mnras/stx2582}{The TESS-HERMES survey data release 1: high-resolution spectroscopy of the TESS southern continuous viewing zone}, \mnras, \textbf{473}, 2004 [\href{https://ui.adsabs.harvard.edu/#abs/2018MNRAS.473.2004S}{61~citations}]

\item[{\color{numcolor}\scriptsize12}] \textbf{Simpson, Jeffrey D.}, De Silva, Gayandhi, Martell, Sarah L., \etal, 2017, \doi{10.1093/mnras/stx2174}{ESO 452-SC11: the lowest mass globular cluster with a potential chemical inhomogeneity}, \mnras, \textbf{472}, 2856 [\href{https://ui.adsabs.harvard.edu/#abs/2017MNRAS.472.2856S}{15~citations}]

\item[{\color{numcolor}\scriptsize11}] \textbf{Simpson, Jeffrey D.}, De Silva, G. M., Martell, S. L., \etal, 2017, \doi{10.1093/mnras/stx1892}{Siriusly, a newly identified intermediate-age Milky Way stellar cluster: a spectroscopic study of Gaia 1}, \mnras, \textbf{471}, 4087 [\href{https://ui.adsabs.harvard.edu/#abs/2017MNRAS.471.4087S}{12~citations}]

\item[{\color{numcolor}\scriptsize10}] Martell, S. L., Sharma, S., Buder, S., \etal\ (incl.\ \textbf{JDS}), 2017, \doi{10.1093/mnras/stw2835}{The GALAH survey: observational overview and Gaia DR1 companion}, \mnras, \textbf{465}, 3203 [\href{https://ui.adsabs.harvard.edu/#abs/2017MNRAS.465.3203M}{122~citations}]

\item[{\color{numcolor}\scriptsize9}] Traven, G., Matijevi{\v{c}}, G., Zwitter, T., \etal\ (incl.\ \textbf{JDS}), 2017, \doi{10.3847/1538-4365/228/2/24}{The Galah Survey: Classification and Diagnostics with t-SNE Reduction of Spectral Information}, The Astrophysical Journal Supplement Series, \textbf{228}, 24 [\href{https://ui.adsabs.harvard.edu/#abs/2017ApJS..228...24T}{36~citations}]

\item[{\color{numcolor}\scriptsize8}] \textbf{Simpson, Jeffrey D.}, Martell, Sarah L., \& Navin, Colin A., 2017, \doi{10.1093/mnras/stw2781}{A broad perspective on multiple abundance populations in the globular cluster NGC 1851}, \mnras, \textbf{465}, 1123 [\href{https://ui.adsabs.harvard.edu/#abs/2017MNRAS.465.1123S}{15~citations}]

\item[{\color{numcolor}\scriptsize7}] Kos, Janez, Lin, Jane, Zwitter, Toma{\v{z}}, \etal\ (incl.\ \textbf{JDS}), 2017, \doi{10.1093/mnras/stw2064}{The GALAH survey: the data reduction pipeline}, \mnras, \textbf{464}, 1259 [\href{https://ui.adsabs.harvard.edu/#abs/2017MNRAS.464.1259K}{46~citations}]

\item[{\color{numcolor}\scriptsize6}] MacLean, B. T., Campbell, S. W., De Silva, G. M., \etal\ (incl.\ \textbf{JDS}), 2016, \doi{10.1093/mnrasl/slw073}{An extreme paucity of second population AGB stars in the `normal' globular cluster M4}, \mnras, \textbf{460} [\href{https://ui.adsabs.harvard.edu/#abs/2016MNRAS.460L..69M}{29~citations}]

\item[{\color{numcolor}\scriptsize5}] \textbf{Simpson, Jeffrey D.}, De Silva, G. M., Bland-Hawthorn, J., \etal, 2016, \doi{10.1093/mnras/stw746}{The GALAH survey: relative throughputs of the 2dF fibre positioner and the HERMES spectrograph from stellar targets}, \mnras, \textbf{459}, 1069 [\href{https://ui.adsabs.harvard.edu/#abs/2016MNRAS.459.1069S}{7~citations}]

\item[{\color{numcolor}\scriptsize4}] Sheinis, Andrew, Anguiano, Borja, Asplund, Martin, \etal\ (incl.\ \textbf{JDS}), 2015, \doi{10.1117/1.JATIS.1.3.035002}{First light results from the High Efficiency and Resolution Multi-Element Spectrograph at the Anglo-Australian Telescope}, Journal of Astronomical Telescopes, Instruments, and Systems, \textbf{1}, 35002 [\href{https://ui.adsabs.harvard.edu/#abs/2015JATIS...1c5002S}{46~citations}]

\item[{\color{numcolor}\scriptsize3}] De Silva, G. M., Freeman, K. C., Bland-Hawthorn, J., \etal\ (incl.\ \textbf{JDS}), 2015, \doi{10.1093/mnras/stv327}{The GALAH survey: scientific motivation}, \mnras, \textbf{449}, 2604 [\href{https://ui.adsabs.harvard.edu/#abs/2015MNRAS.449.2604D}{404~citations}]

\item[{\color{numcolor}\scriptsize2}] \textbf{Simpson, Jeffrey D.}, \& Cottrell, P. L., 2013, \doi{10.1093/mnras/stt857}{Spectral matching for abundances of 848 stars of the giant branches of the globular cluster {\ensuremath{\omega}} Centauri}, \mnras, \textbf{433}, 1892 [\href{https://ui.adsabs.harvard.edu/#abs/2013MNRAS.433.1892S}{11~citations}]

\item[{\color{numcolor}\scriptsize1}] \textbf{Simpson, Jeffrey D.}, Cottrell, P. L., \& Worley, C. C., 2012, \doi{10.1111/j.1365-2966.2012.22012.x}{Spectral matching for abundances and clustering analysis of stars on the giant branches of {\ensuremath{\omega}} Centauri}, \mnras, \textbf{427}, 1153 [\href{https://ui.adsabs.harvard.edu/#abs/2012MNRAS.427.1153S}{13~citations}]
\end{list}

\subsection*{In submission}
\begin{list}{}{\cvlist}
\item[{\color{numcolor}\scriptsize5}] Bland-Hawthorn, Joss, Sharma, Sanjib, Tepper-Garcia, Thor, \etal\ (incl.\ \textbf{JDS}), 2018, The GALAH survey and Gaia DR2: dissecting the stellar disc's phase space by age, action, chemistry and location (\arxiv{1809.02658})

\item[{\color{numcolor}\scriptsize4}] Khanna, Shourya, Sharma, Sanjib, Bland-Hawthorn, Joss, \etal\ (incl.\ \textbf{JDS}), 2018, The GALAH Survey: Velocity fluctuations in the Milky Way using red clump giants (\arxiv{1804.07217})

\item[{\color{numcolor}\scriptsize3}] Buder, S., Lind, K., Ness, M. K., \etal\ (incl.\ \textbf{JDS}), 2018, The GALAH survey: An abundance, age, and kinematic inventory of the solar neighbourhood made with TGAS (\arxiv{1804.05869}) [\href{https://ui.adsabs.harvard.edu/#abs/2018arXiv180405869B}{4~citations}]

\item[{\color{numcolor}\scriptsize2}] \textbf{Simpson, Jeffrey D.}, Martell, Sarah L., Da Costa, Gary, \etal, 2018, The GALAH survey: Co-orbiting stars and chemical tagging (\arxiv{1804.05894})

\item[{\color{numcolor}\scriptsize1}] \textbf{Simpson, Jeffrey D.}, Stello, Dennis, Sharma, Sanjib, \etal, 2018, The GALAH and TESS-HERMES surveys: high-resolution spectroscopy of luminous supergiants in the Magellanic Clouds and Bridge (\arxiv{1804.05900})
\end{list}

\subsection*{Invited conference talks}
\begin{list}{}{\cvlist}
\item[{\color{numcolor}\scriptsize1}] 2019: Stars, Streams, Clusters Oh My, at \textit{Stars In Melbourne}. Melbourne, Australia.

\end{list}

\subsection*{Competitive observing proposals}

\textbf{Anglo-Australian Telescope}
\begin{innerlist}
\item Co-I: A Comprehensive Spectroscopic Survey of Southern Stellar Streams (9 nights) \hfill{21A}
\item Co-I: The GALAH Survey: Phase 2 (155 nights) \hfill{20B}
\item Co-I: The astrophysical origins of spectro-seismology (15 nights) \hfill{20B}
\item Co-I: The K2-HERMES follow-up program (13 nights) \hfill{20B}
\item Co-I: The Southern Stellar Stream Spectroscopic Survey (14 nights) \hfill{20B}
\item Co-I: The K2-HERMES follow-up program (15.5 nights) \hfill{20A}
\item Co-I: The Southern Stellar Stream Spectroscopic Survey (13 nights) \hfill{20A}
\item Co-I: Tracing the metal-poor tail of the inner Galaxy with the Pristine survey (4.5 nights) \hfill{20A}
\item Co-I: The HERMES K2 followup program (10 nights) \hfill{19B}
\item \textbf{PI}: Chemical tagging between stellar streams and globular clusters (3 nights) \hfill{19B}
\item Co-I: The HERMES K2 followup program (10 nights) \hfill{19B}
\item Co-I: How many extremely metal-poor stars in the Milky Way are on disk orbits? (3 nights) \hfill{19B}
\item Co-I: The GALAH Survey: Phase 2 (41 nights) \hfill{19A}
\item Co-I: The Galaxy's Dark Side: Dynamical Studies with the Southern Stellar Stream Spectroscopic Survey (10 nights) \hfill{19A}
\item Co-I: Hierarchical star formation in Ori OB1 (4 nights) \hfill{19A}
\item Co-I: Dynamical Studies of DES Stellar Streams (10 nights) \hfill{18B}
\item Co-I: The HERMES-TESS program (8 nights) \hfill{18B}
\item Co-I: Open clusters with HERMES (5 nights) \hfill{18A}
\item Co-I: Open clusters with HERMES (13 nights) \hfill{17B}
\item Co-I: How Extended is the Stellar Envelope of NGC5694? (6 hours) \hfill{17A}
\item Co-I: The GALAH Survey: Phase 2 (35 nights/semester) \hfill{17A--17B}
\item Co-I: The HERMES K2-follow-up program \\ (12 nights/semester)  \hfill{16A--17B}
\item \textbf{PI}: Probing the low mass regime of globular clusters (6 hours) \hfill{16A}
\item Co-I: The GALAH Survey (35 nights/semester) \hfill{15A--16B}
\end{innerlist}
\vspace{0.1in}
\textbf{Keck Observatory}
\begin{innerlist}
\item \textbf{PI}: ESO452: Exploring self-enrichment in low mass stellar clusters\\ (0.5 nights)  \hfill{17A}
\end{innerlist}
\vspace{0.1in}
\textbf{Magellan Telescopes}
\begin{innerlist}
\item \textbf{PI}: Chemical abundances of a faint, metal-poor globular cluster\\ (1 night)  \hfill{19A}
\end{innerlist}

\subsection*{Conference Proceedings}
\begin{list}{}{\cvlist}
\item[{\color{numcolor}\scriptsize4}] Edgar, Michael L., Zhelem, Ross, Waller, Lewis, \etal\ (incl.\ \textbf{JDS}), 2018, \doi{10.1117/12.2307305}{Radioactive emission from high-index,optical glasses and atypical effects on CCDs}, Advances in Optical and Mechanical Technologies for Telescopes and Instrumentation III, \textbf{10706}, 1070633 [\href{https://ui.adsabs.harvard.edu/#abs/2018SPIE10706E..33E}{2~citations}]

\item[{\color{numcolor}\scriptsize3}] Sheinis, Andrew, Barden, Sam, Birchall, Michael, \etal\ (incl.\ \textbf{JDS}), 2014, \doi{10.1117/12.2055595}{First light results from the Hermes spectrograph at the AAT}, Ground-based and Airborne Instrumentation for Astronomy V, \textbf{9147} [\href{https://ui.adsabs.harvard.edu/#abs/2014SPIE.9147E..0YS}{15~citations}]

\item[{\color{numcolor}\scriptsize2}] \textbf{Simpson, Jeffrey D.}, 2012, \doi{10.22323/1.146.0232}{Carbon, nitrogen and barium abundances of giant branch stars of {\ensuremath{\alpha}} Centauri using spectral matching}, Nuclei in the Cosmos (NIC XII), 232

\item[{\color{numcolor}\scriptsize1}] Worley, C., Cottrell, P., \& \textbf{Simpson, Jeffrey D.}, 2010, \doi{10.22323/1.100.0201}{Neutron-capture element abundances in the globular clusters: 47 Tuc, NGC 6388 and NGC 362}, Nuclei in the Cosmos, 201
\end{list}

\subsection*{Contributed conference talks}
\begin{list}{}{\cvlist}
\item[{\color{numcolor}\scriptsize14}] 2018: A very nitrogen-rich star in the very low-mass, very metal-poor cluster ESO280-SC06, at Survival of Dense Star Clusters in the Milky Way System. Heidelberg, Germany.
\item[{\color{numcolor}\scriptsize13}] 2018: Flying the nest to the Magellanic Clouds and Bridge with GALAH and TESS-HERMES~, at  ASA Annual Scientific Meeting. Melbourne, Australia.
\item[{\color{numcolor}\scriptsize12}] 2018: Pushing the envelope on globular clusters, at  ASA Annual Scientific Meeting. Melbourne, Australia.
\item[{\color{numcolor}\scriptsize11}] 2017: The GALAH survey: Discovery of dissolving star clusters, at  Surveying the Cosmos, The Science From Massively Multiplexed Surveys. Sydney, Australia.
\item[{\color{numcolor}\scriptsize10}] 2017: What happened to the horizontal branch of ESO280-SC06? In Stars in Sydney. Sydney, Australia.
\item[{\color{numcolor}\scriptsize9}] 2017: The GALAH survey: Co-orbiting stars and chemical tagging, at  Celebration of CEMP \& Gala of GALAH workshop. Melbourne, Australia.
\item[{\color{numcolor}\scriptsize8}] 2017: What happened to the horizontal branch of ESO280-SC06? In Australian Institute of Physics Summer Meeting 2017. Sydney, Australia.
\item[{\color{numcolor}\scriptsize7}] 2016: Probing the low-mass regime of globular clusters, at  Multiple populations in globular clusters: Where do we stand? Sexten, Italy.
\item[{\color{numcolor}\scriptsize6}] 2016: Tips and tools to work with reduced data, at  ITSO/AAO Observational Techniques Workshop. Sydney, Australia.
\item[{\color{numcolor}\scriptsize5}] 2015: Searching extra-tidal stars of globular clusters with the GALAH survey, at  Multiwavelength Dissection of Galaxies. Sydney, Australia.
\item[{\color{numcolor}\scriptsize4}] 2014: C+N+O abundance of evolved stars of NGC1851, at  Bolton Symposium. Sydney, Australia.
\item[{\color{numcolor}\scriptsize3}] 2013: Spectral matching for elemental abundances of evolved stars of globular clusters, at  The Origin of Cosmic Elements. Barcelona, Spain.
\item[{\color{numcolor}\scriptsize2}] 2012: Carbon, nitrogen and barium abundances of giant branch stars of $\omega$ Centauri using spectral matching. Cairns, Australia.
\item[{\color{numcolor}\scriptsize1}] 2011: Stellar parameters and barium abundances in $\omega$ Centauri GB by spectral matching, at  6th Stromlo Symposium on IFU Science in Australia. Canberra, Australia.

\end{list}

\subsection*{Conference Activity}
\begin{innerlist}
	\item Chaired organizing committees for the 2017 Southern Cross Astrophysics Conference on ``Surveying the Cosmos, The Science From Massively Multiplexed Surveys''
	\item Local organizing committee for LSST@Asia (2019)
\end{innerlist}


\subsection*{Service To Profession}
\begin{innerlist}
\item Postdoctoral representative to faculty committee (2019)
\item Member of Anglo-Australian Telescope Users' Committee (2018--2021)
\item Referee for articles in PASA, A\&A, and MNRAS
\item Referee for research funding proposal for Polish National Science Centre
\end{innerlist}

\subsection*{References Available to Contact}

\href
{https://www.physics.unsw.edu.au/staff/sarah-martell}
{\textbf{Sarah Martell}}
%
\begin{innerlist}
   \item \href{mailto:s.martell@unsw.edu.au}{s.martell@unsw.edu.au}
   \item (02) 9385 6694
   \item School of Physics, The University of New South Wales, Sydney NSW 2052, Australia
\end{innerlist}

\halfblankline

\href
{https://rsaa.anu.edu.au/people/academics/dr-chris-lidman}
{\textbf{Chris Lidman}}
%
\begin{innerlist}
   \item \href{mailto:christopher.lidman@anu.edu.au}{christopher.lidman@anu.edu.au}
   \item (02) 6125 0238
   \item Research School of Astronomy \& Astrophysics Mount Stromlo Observatory Cotter Road Weston Creek, ACT 2611 Australia
\end{innerlist}

\halfblankline

\href
{https://rsaa.anu.edu.au/people/academics/dr-chris-lidman}
{\textbf{Gary Da Costa}}
\begin{innerlist}
  \item \href{mailto:gary.dacosta@anu.edu.au}{gary.dacosta@anu.edu.au}
  \item (02) 6125 8913
   \item Research School of Astronomy \& Astrophysics Mount Stromlo Observatory Cotter Road Weston Creek, ACT 2611 Australia
\end{innerlist}

\end{document}

%%%%%%%%%%%%%%%%%%%%%%%%%% End CV Document %%%%%%%%%%%%%%%%%%%%%%%%%%%%%

%----------------------------------------------------------------------%
% The following is copyright and licensing information for
% redistribution of this LaTeX source code; it also includes a liability
% statement. If this source code is not being redistributed to others,
% it may be omitted. It has no effect on the function of the above code.
%----------------------------------------------------------------------%
% Copyright (c) 2007, 2008, 2009, 2010, 2011 by Theodore P. Pavlic
%
% Unless otherwise expressly stated, this work is licensed under the
% Creative Commons Attribution-Noncommercial 3.0 United States License. To
% view a copy of this license, visit
% http://creativecommons.org/licenses/by-nc/3.0/us/ or send a letter to
% Creative Commons, 171 Second Street, Suite 300, San Francisco,
% California, 94105, USA.
%
% THE SOFTWARE IS PROVIDED "AS IS", WITHOUT WARRANTY OF ANY KIND, EXPRESS
% OR IMPLIED, INCLUDING BUT NOT LIMITED TO THE WARRANTIES OF
% MERCHANTABILITY, FITNESS FOR A PARTICULAR PURPOSE AND NONINFRINGEMENT.
% IN NO EVENT SHALL THE AUTHORS OR COPYRIGHT HOLDERS BE LIABLE FOR ANY
% CLAIM, DAMAGES OR OTHER LIABILITY, WHETHER IN AN ACTION OF CONTRACT,
% TORT OR OTHERWISE, ARISING FROM, OUT OF OR IN CONNECTION WITH THE
% SOFTWARE OR THE USE OR OTHER DEALINGS IN THE SOFTWARE.
%----------------------------------------------------------------------%
