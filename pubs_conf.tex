\item[{\color{numcolor}\scriptsize15}] 2019: (Poster) The Fimbulthul stellar stream was tidally stripped from the globular cluster $\omega$ Centauri, at ASA Annual Scientific Meeting. Brisbane, Australia. (Winner of Best Poster)
\item[{\color{numcolor}\scriptsize15}] 2019: Mapping stellar streams with LSST, at LSST@Asia. Sydney, Australia.
\item[{\color{numcolor}\scriptsize14}] 2018: A very nitrogen-rich star in the very low-mass, very metal-poor cluster ESO280-SC06, at Survival of Dense Star Clusters in the Milky Way System. Heidelberg, Germany.
\item[{\color{numcolor}\scriptsize13}] 2018: Flying the nest to the Magellanic Clouds and Bridge with GALAH and TESS-HERMES~, at ASA Annual Scientific Meeting. Melbourne, Australia.
\item[{\color{numcolor}\scriptsize12}] 2018: Pushing the envelope on globular clusters, at  ASA Annual Scientific Meeting. Melbourne, Australia.
\item[{\color{numcolor}\scriptsize11}] 2017: The GALAH survey: Discovery of dissolving star clusters, at  Surveying the Cosmos, The Science From Massively Multiplexed Surveys. Sydney, Australia.
\item[{\color{numcolor}\scriptsize10}] 2017: What happened to the horizontal branch of ESO280-SC06? In Stars in Sydney. Sydney, Australia.
\item[{\color{numcolor}\scriptsize9}] 2017: The GALAH survey: Co-orbiting stars and chemical tagging, at  Celebration of CEMP \& Gala of GALAH workshop. Melbourne, Australia.
\item[{\color{numcolor}\scriptsize8}] 2017: What happened to the horizontal branch of ESO280-SC06? In Australian Institute of Physics Summer Meeting 2017. Sydney, Australia.
\item[{\color{numcolor}\scriptsize7}] 2016: Probing the low-mass regime of globular clusters, at  Multiple populations in globular clusters: Where do we stand? Sexten, Italy.
\item[{\color{numcolor}\scriptsize6}] 2016: Tips and tools to work with reduced data, at  ITSO/AAO Observational Techniques Workshop. Sydney, Australia.
\item[{\color{numcolor}\scriptsize5}] 2015: Searching extra-tidal stars of globular clusters with the GALAH survey, at  Multiwavelength Dissection of Galaxies. Sydney, Australia.
\item[{\color{numcolor}\scriptsize4}] 2014: C+N+O abundance of evolved stars of NGC1851, at  Bolton Symposium. Sydney, Australia.
\item[{\color{numcolor}\scriptsize3}] 2013: Spectral matching for elemental abundances of evolved stars of globular clusters, at The Origin of Cosmic Elements. Barcelona, Spain.
\item[{\color{numcolor}\scriptsize2}] 2012: Carbon, nitrogen and barium abundances of giant branch stars of $\omega$ Centauri using spectral matching, at Nuclei in the Cosmos. Cairns, Australia.
\item[{\color{numcolor}\scriptsize1}] 2011: Stellar parameters and barium abundances in $\omega$ Centauri GB by spectral matching, at  6th Stromlo Symposium on IFU Science in Australia. Canberra, Australia.
